\documentclass[journal,10pt,twocolumn]{article}
\usepackage{graphicx}
\usepackage[margin=0.5in]{geometry}
\usepackage[cmex10]{amsmath}
\usepackage{array}
\usepackage{booktabs}
\usepackage{mathtools}
\title{\textbf{Optimization Assignment - 2}}
\author{Mohamed Hamdan}
\date{September 2022}


\providecommand{\norm}[1]{\left\lVert#1\right\rVert}
\providecommand{\abs}[1]{\left\vert#1\right\vert}
\let\vec\mathbf
\newcommand{\myvec}[1]{\ensuremath{\begin{pmatrix}#1\end{pmatrix}}}
\newcommand{\mydet}[1]{\ensuremath{\begin{vmatrix}#1\end{vmatrix}}}
\providecommand{\brak}[1]{\ensuremath{\left(#1\right)}}
\providecommand{\lbrak}[1]{\ensuremath{\left(#1\right.}}
\providecommand{\rbrak}[1]{\ensuremath{\left.#1\right)}}
\providecommand{\sbrak}[1]{\ensuremath{{}\left[#1\right]}}

\begin{document}

\maketitle
\paragraph{\textit{Problem Statement} - An open topped box is to be constructed by removing equal squares from each corner of a 3 meter by 8 meter rectangular sheet of aluminium and folding up the sides. Find the volume of the largest such box}

\section*{\large Solution}
\subsection*{\normalsize Monomial approximation}
Let $h$ be the side of the square removed. Let $l$ and $b$ be the length and breadth of the box. Due to the construction, $h$ is the height of the box. The volume of the box can be expressed as
\begin{align}
	V = lbh
\end{align}
where
\begin{align}
	\label{eq:con_eq1}
	l = 8 - 2h\\
	\label{eq:con_eq2}
	b = 3 - 2h
\end{align}
Since l and h are positive, from \eqref{eq:con_eq1} and \eqref{eq:con_eq2} we get
\begin{align}
	h <= \min(4, 1.5)\\
	\implies h <= 1.5
	\label{eq:lt_eq1}
\end{align}
The problem can be formulated as
\begin{align}
	\label{eq:prob_std}
	P = \max_{l,b,h}lbh\\
	\label{eq:con_eq_st1}
	l + 2h = 8\\
	\label{eq:con_eq_st2}
	b + 2h = 3\\
	\label{eq:lt_eq_std}
	h <= 1.5
\end{align}
The above formulation is not a geometric programming problem since the equality constraints \eqref{eq:con_eq_st1} and \eqref{eq:con_eq_st2} are posynomials. To convert the problem into a geometric programming problem, we approximate the posynomials in the equality constraints as monomials using
\begin{align}
	\label{eq:mono_approx}
	f(w) \approx f(x)\prod_{i=1}^{n}\brak{\frac{w_i}{x_i}}^{a_i}
\end{align}
where
\begin{align}
	a_i = \frac{x_i}{f(x)}\frac{\partial{f}}{\partial{x_i}}
\end{align}
Approximating \eqref{eq:con_eq_st1} and \eqref{eq:con_eq_st2} as $f(l,b,h)$ and $g(l,b,h)$ respectively using \eqref{eq:mono_approx}, the problem is expressed as
\begin{align}
	P = \max_{l,b,h}lbh\\
	f(l,b,h) = 8\\
	g(l,b,h) = 3\\
	h >= \frac{h_k}{1+\Delta}\\
	h <= \min((1+\Delta)h_k, 1.5)
\end{align}
Where $h_k$ is the current guess of $h$ and $\Delta$ is the parameter to determine the trust region around $h_k$ where the approximation in \eqref{eq:mono_approx} holds valid.
The above formulation can be iterated till the problem converges at a local maximum. Taking $n = 100$ and $\Delta = 0.005$, the optimal solution obtained using cvxpy is
\begin{align}
	\boxed{V_{max} = 7.407485}\\
	\boxed{h = 0.664331}
\end{align}

\subsection*{\normalsize Gradient descent}
Let $x$ be the side of each square removed. The volume of the box can be expressed as
\begin{align}
	\label{eq:vol_varx}
	f(x) = (8-2x)(3-2x)x\\
	\implies f(x) = 4x^3-22x^2+24x
	\label{eq:cubic_exp}
\end{align}
The polynomial in \eqref{eq:cubic_exp} has roots at $x = 0$, $x = 1.5$ and $x = 4$. Since the coefficient of $x^3$ is positive and the roots are distinct, it can be concluded that there exists a local maximum between $x = 0$ and $x = 1.5$. This can be seen in Figure \ref{fig:graph_fx}.
Using gradient ascent method we can find its maxima in the interval $\mathbf{[}0,1.5\mathbf{]}$,
    \begin{align}
        x_{n+1} &= x_n + \alpha \nabla f(x_n) \\
        \implies x_{n+1} &= x_n + \alpha \brak{12x_n^2-44x_n+24}
    \end{align}
    
Taking $x_0=0.5,\alpha=0.001$ and precision = 0.00000001, values obtained using python are:
    
    \begin{align}
        \boxed{\text{Maxima} = 7.407407}\\
        \boxed{\text{Maxima Point} = 0.666666}
    \end{align}

\begin{figure}[t]
	\centering
	\includegraphics[width=1\columnwidth]{figs/fig1.pdf}
	\caption{Graph of $f(x)$ in \eqref{eq:cubic_exp}}
	\label{fig:graph_fx}
\end{figure}

\end{document}
