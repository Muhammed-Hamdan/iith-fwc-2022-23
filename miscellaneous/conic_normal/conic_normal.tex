\documentclass[journal,10pt,twocolumn]{article}
\usepackage{graphicx}
\usepackage[margin=0.5in]{geometry}
\usepackage[cmex10]{amsmath}
\usepackage{array}
\usepackage{booktabs}
\usepackage{mathtools}
\title{\textbf{Normal(s) to a conic from a point}}

\providecommand{\norm}[1]{\left\lVert#1\right\rVert}
\providecommand{\abs}[1]{\left\vert#1\right\vert}
\let\vec\mathbf
\newcommand{\myvec}[1]{\ensuremath{\begin{pmatrix}#1\end{pmatrix}}}
\newcommand{\mydet}[1]{\ensuremath{\begin{vmatrix}#1\end{vmatrix}}}
\providecommand{\brak}[1]{\ensuremath{\left(#1\right)}}
\providecommand{\lbrak}[1]{\ensuremath{\left(#1\right.}}
\providecommand{\rbrak}[1]{\ensuremath{\left.#1\right)}}
\providecommand{\sbrak}[1]{\ensuremath{{}\left[#1\right]}}

\begin{document}

\maketitle
\paragraph{\textit{Problem Statement} - To find the equation of all possible normals to a conic from a point}\

Let the point from which normals are drawn be $\vec{h}$. Then, the equation of the normal can be written as
\begin{align}
	\vec{x} = \vec{h} + \lambda\vec{m}
	\label{eq:normal_chord}
\end{align}
Say the point of intersection of \eqref{eq:normal_chord} with the conic is $\vec{q}$. A tangent drawn at $\vec{q}$ satisfies the equation
\begin{align}
	\label{eq:tangency_condition}
	\vec{n}^\top(\vec{Vq}+\vec{u}) = 0
\end{align}
Where $\vec{n}$ is the direction vector of the tangent and is perpendicular to $\vec{m}$ in \eqref{eq:normal_chord}.\\\\
In general, the parameter values for points of intersection of a line given by \eqref{eq:normal_chord} with a conic is given by
{\tiny
\begin{multline}
\lambda_i = \frac{1}
{
\vec{m}^T\vec{V}\vec{m}
}
\lbrak{-\vec{m}^T\brak{\vec{V}\vec{h}+\vec{u}}}
\\
\pm
\rbrak{\sqrt{
\sbrak{
\vec{m}^T\brak{\vec{V}\vec{h}+\vec{u}}
}^2
-
\brak
{
\vec{h}^T\vec{V}\vec{h} + 2\vec{u}^T\vec{h} +f
}
\brak{\vec{m}^T\vec{V}\vec{m}}
}
}
\label{eq:tangent_roots}
\end{multline}
}
Using \eqref{eq:tangent_roots} and \eqref{eq:normal_chord}, the intersection point $\vec{q}$ can be written as
\begin{align}
	\label{eq:point_of_tangency}
	\vec{q} = \vec{h} + \lambda_i\vec{m}
\end{align}
Substituting \eqref{eq:point_of_tangency} in \eqref{eq:tangency_condition},
\begin{align}
	\label{eq:normal_simp_1}
	\vec{n}^\top(\vec{V}(\vec{h}+\lambda_i\vec{m})+\vec{u}) = 0\\
	\label{eq:normal_simp_2}
	\implies \lambda_i\vec{n}^\top\vec{V}\vec{m} = -\vec{n}^\top(\vec{Vh}+\vec{u})
\end{align}
Substituting value of $\lambda_i$ from \eqref{eq:tangent_roots} in \eqref{eq:normal_simp_2}
{\tiny
\begin{multline}
	\frac{1}{\vec{m}^\top\vec{V}\vec{m}}\lbrak{-\vec{m}^\top\brak{\vec{Vh}+\vec{u}}} \\ 
	\pm \rbrak{\sqrt{\sbrak{\vec{m}^T\brak{\vec{V}\vec{h}+\vec{u}}}^2-\brak{\vec{h}^T\vec{V}\vec{h} + 2\vec{u}^T\vec{h} +f}\brak{\vec{m}^T\vec{V}\vec{m}}}}\vec{n}^\top\vec{V}\vec{m} \\
	= -\vec{n}^\top\brak{\vec{Vh}+\vec{u}}
	\label{eq:normal_simp_3}
\end{multline}
}
Rearranging the terms,
{\tiny
\begin{multline}
	\pm \sqrt{\sbrak{\vec{m}^T\brak{\vec{V}\vec{h}+\vec{u}}}^2-\brak{\vec{h}^T\vec{V}\vec{h} + 2\vec{u}^T\vec{h} +f}\brak{\vec{m}^T\vec{V}\vec{m}}} \\ = \brak{\vec{Vh}+\vec{u}}^\top\brak{\brak{\vec{n}^\top\vec{V}\vec{m}}\vec{m}-\brak{\vec{m}^\top\vec{V}\vec{m}}\vec{n}}
\end{multline}
}
Squaring on both sides
{\tiny
\begin{multline}
	\sbrak{\vec{m}^T\brak{\vec{V}\vec{h}+\vec{u}}}^2-\brak{\vec{h}^T\vec{V}\vec{h} + 2\vec{u}^T\vec{h} +f}\brak{\vec{m}^T\vec{V}\vec{m}} \\ = \sbrak{\brak{\vec{Vh}+\vec{u}}^\top\brak{\brak{\vec{n}^\top\vec{V}\vec{m}}\vec{m}-\brak{\vec{m}^\top\vec{V}\vec{m}}\vec{n}}}^2
	\label{eq:normal_solution}
\end{multline}
}
If $\vec{n}$ is take as $\myvec{1 \\ \mu}$, then $\vec{m}$ is $\myvec{\mu \\ -1}$. Substituting these values in \eqref{eq:normal_solution} and solving for $\mu$, the different possible normals passing through $\vec{h}$ are obtained.
\end{document}
