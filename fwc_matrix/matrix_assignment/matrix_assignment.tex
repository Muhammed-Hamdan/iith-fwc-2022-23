\documentclass[journal,10pt,twocolumn]{article}
\usepackage{graphicx}
\usepackage[margin=0.5in]{geometry}
\usepackage{amsmath}
\title{\textbf{Matrix Assignment}}
\author{Mohamed Hamdan}
\date{September 2022}

\begin{document}
\maketitle
\paragraph{\textit{Problem Statement} - ABC is a triangle right angled at C. A line through the mid-point M of hypotenuse AB and parallel to BC intersects AC at D. Show that}
\begin{enumerate}
	\item \textbf{D is the mid-point of AC}
	\item \textbf{MD $\perp$ AC}
	\item \textbf{CM = MA = $\frac{1}{2}$AB}
\end{enumerate}

\section*{Solution}
\subsection*{Part 1}
Given MN is $\parallel$ to BC, hence\\
\begin{equation}
\frac{AD}{DC} = \frac{AM}{MB}
\label{eq-1}
\end{equation}
Since M is the mid-point of AB, $\frac{AM}{MB} = 1$. Substituting in (\ref{eq-1}), we get\\
\begin{equation}
AD = DC
\label{eq-2}
\end{equation}
Therefore, D is the midpoint of AC.
\subsection*{Part 2}
Since ABC is right angled at C,
\begin{equation}
(\boldsymbol{C-A})^T(\boldsymbol{C-B}) = 0	
\label{eq-3}
\end{equation}
Given that MD is parallel to BC, so
\begin{equation}
(\boldsymbol{C-B}) = \lambda(\boldsymbol{M-D})
\label{eq-4}
\end{equation}
Substituting (\ref{eq-4}) in (\ref{eq-3}) and dividing by $\lambda$, we get
\begin{equation}
(\boldsymbol{C-A})^T(\boldsymbol{M-B}) = 0	
\label{eq-5}
\end{equation}
From (\ref{eq-5}) it can be concluded that MD $\perp$ AC.
\subsection*{Part 3}
Let
\begin{eqnarray}
	\boldsymbol{M-D} = \boldsymbol{p}\\
	\boldsymbol{C-D} = \boldsymbol{q}\\
	\boldsymbol{A-D} = \boldsymbol{r}
\end{eqnarray}
Then vectors along CM and AM can be written as
\begin{eqnarray}
	\boldsymbol{C-M} = \boldsymbol{q-p}\\
	\boldsymbol{A-M} = \boldsymbol{r-p}
\end{eqnarray}
The magnitudes of CM and AM are therefore
\begin{eqnarray}
	||\boldsymbol{C-M}|| = ||\boldsymbol{q-p}|| = (\boldsymbol{q-p})^T(\boldsymbol{q-p})\\
	||\boldsymbol{A-M}|| = ||\boldsymbol{r-p}|| = (\boldsymbol{r-p})^T(\boldsymbol{r-p})\\
\end{eqnarray}
Upon expanding the vector products, the terms $\boldsymbol{q}^T\boldsymbol{p}$ and $\boldsymbol{r}^T\boldsymbol{p}$ evaluate to 0 (from eq.\ref{eq-5}). From eq.\ref{eq-2}, $$||\boldsymbol{q}|| = ||\boldsymbol{r}||$$. Therefore,
\begin{eqnarray}
||\boldsymbol{C-M}|| = ||\boldsymbol{q}|| + ||\boldsymbol{p}||\\
	||\boldsymbol{A-M}|| = ||\boldsymbol{r}|| + ||\boldsymbol{p}|| = ||\boldsymbol{q}|| + ||\boldsymbol{p}||
\end{eqnarray}
Equating () and (), we get $$AM = CM$$
Since given that M is midpoint of AB, $$AM = \frac{1}{2}AB$$
Equating () and (), the result is proved.
\section*{Construction}
The input parameters are the lengths a and c.
\begin{tabular}{|c|c|c|}
	\textbf{Symbol}&\textbf{Value}&\textbf{Description}\\
	a&4&BC\\
	c&5&AC\\
	k&1&$\frac{AM}{MB} = \frac{AD}{DC}$\\
	$\theta$&arctan($\frac{c}{a}$)&$\angle$B\\
	A&$\sqrt{a^2+c^2}%
	\begin{pmatrix}
		cos\theta\\
		sin\theta\\
	\end{pmatrix}$%
	&Point A
\end{tabular}
\end{document}
